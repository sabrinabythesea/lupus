\documentclass[]{article}
\usepackage{lmodern}
\usepackage{amssymb,amsmath}
\usepackage{ifxetex,ifluatex}
\usepackage{fixltx2e} % provides \textsubscript
\ifnum 0\ifxetex 1\fi\ifluatex 1\fi=0 % if pdftex
  \usepackage[T1]{fontenc}
  \usepackage[utf8]{inputenc}
\else % if luatex or xelatex
  \ifxetex
    \usepackage{mathspec}
  \else
    \usepackage{fontspec}
  \fi
  \defaultfontfeatures{Ligatures=TeX,Scale=MatchLowercase}
\fi
% use upquote if available, for straight quotes in verbatim environments
\IfFileExists{upquote.sty}{\usepackage{upquote}}{}
% use microtype if available
\IfFileExists{microtype.sty}{%
\usepackage{microtype}
\UseMicrotypeSet[protrusion]{basicmath} % disable protrusion for tt fonts
}{}
\usepackage[margin=1in]{geometry}
\usepackage{hyperref}
\hypersetup{unicode=true,
            pdftitle={Lupus Symptoms on Twitter},
            pdfauthor={Sabrina Kent},
            pdfborder={0 0 0},
            breaklinks=true}
\urlstyle{same}  % don't use monospace font for urls
\usepackage{color}
\usepackage{fancyvrb}
\newcommand{\VerbBar}{|}
\newcommand{\VERB}{\Verb[commandchars=\\\{\}]}
\DefineVerbatimEnvironment{Highlighting}{Verbatim}{commandchars=\\\{\}}
% Add ',fontsize=\small' for more characters per line
\usepackage{framed}
\definecolor{shadecolor}{RGB}{248,248,248}
\newenvironment{Shaded}{\begin{snugshade}}{\end{snugshade}}
\newcommand{\AlertTok}[1]{\textcolor[rgb]{0.94,0.16,0.16}{#1}}
\newcommand{\AnnotationTok}[1]{\textcolor[rgb]{0.56,0.35,0.01}{\textbf{\textit{#1}}}}
\newcommand{\AttributeTok}[1]{\textcolor[rgb]{0.77,0.63,0.00}{#1}}
\newcommand{\BaseNTok}[1]{\textcolor[rgb]{0.00,0.00,0.81}{#1}}
\newcommand{\BuiltInTok}[1]{#1}
\newcommand{\CharTok}[1]{\textcolor[rgb]{0.31,0.60,0.02}{#1}}
\newcommand{\CommentTok}[1]{\textcolor[rgb]{0.56,0.35,0.01}{\textit{#1}}}
\newcommand{\CommentVarTok}[1]{\textcolor[rgb]{0.56,0.35,0.01}{\textbf{\textit{#1}}}}
\newcommand{\ConstantTok}[1]{\textcolor[rgb]{0.00,0.00,0.00}{#1}}
\newcommand{\ControlFlowTok}[1]{\textcolor[rgb]{0.13,0.29,0.53}{\textbf{#1}}}
\newcommand{\DataTypeTok}[1]{\textcolor[rgb]{0.13,0.29,0.53}{#1}}
\newcommand{\DecValTok}[1]{\textcolor[rgb]{0.00,0.00,0.81}{#1}}
\newcommand{\DocumentationTok}[1]{\textcolor[rgb]{0.56,0.35,0.01}{\textbf{\textit{#1}}}}
\newcommand{\ErrorTok}[1]{\textcolor[rgb]{0.64,0.00,0.00}{\textbf{#1}}}
\newcommand{\ExtensionTok}[1]{#1}
\newcommand{\FloatTok}[1]{\textcolor[rgb]{0.00,0.00,0.81}{#1}}
\newcommand{\FunctionTok}[1]{\textcolor[rgb]{0.00,0.00,0.00}{#1}}
\newcommand{\ImportTok}[1]{#1}
\newcommand{\InformationTok}[1]{\textcolor[rgb]{0.56,0.35,0.01}{\textbf{\textit{#1}}}}
\newcommand{\KeywordTok}[1]{\textcolor[rgb]{0.13,0.29,0.53}{\textbf{#1}}}
\newcommand{\NormalTok}[1]{#1}
\newcommand{\OperatorTok}[1]{\textcolor[rgb]{0.81,0.36,0.00}{\textbf{#1}}}
\newcommand{\OtherTok}[1]{\textcolor[rgb]{0.56,0.35,0.01}{#1}}
\newcommand{\PreprocessorTok}[1]{\textcolor[rgb]{0.56,0.35,0.01}{\textit{#1}}}
\newcommand{\RegionMarkerTok}[1]{#1}
\newcommand{\SpecialCharTok}[1]{\textcolor[rgb]{0.00,0.00,0.00}{#1}}
\newcommand{\SpecialStringTok}[1]{\textcolor[rgb]{0.31,0.60,0.02}{#1}}
\newcommand{\StringTok}[1]{\textcolor[rgb]{0.31,0.60,0.02}{#1}}
\newcommand{\VariableTok}[1]{\textcolor[rgb]{0.00,0.00,0.00}{#1}}
\newcommand{\VerbatimStringTok}[1]{\textcolor[rgb]{0.31,0.60,0.02}{#1}}
\newcommand{\WarningTok}[1]{\textcolor[rgb]{0.56,0.35,0.01}{\textbf{\textit{#1}}}}
\usepackage{graphicx,grffile}
\makeatletter
\def\maxwidth{\ifdim\Gin@nat@width>\linewidth\linewidth\else\Gin@nat@width\fi}
\def\maxheight{\ifdim\Gin@nat@height>\textheight\textheight\else\Gin@nat@height\fi}
\makeatother
% Scale images if necessary, so that they will not overflow the page
% margins by default, and it is still possible to overwrite the defaults
% using explicit options in \includegraphics[width, height, ...]{}
\setkeys{Gin}{width=\maxwidth,height=\maxheight,keepaspectratio}
\IfFileExists{parskip.sty}{%
\usepackage{parskip}
}{% else
\setlength{\parindent}{0pt}
\setlength{\parskip}{6pt plus 2pt minus 1pt}
}
\setlength{\emergencystretch}{3em}  % prevent overfull lines
\providecommand{\tightlist}{%
  \setlength{\itemsep}{0pt}\setlength{\parskip}{0pt}}
\setcounter{secnumdepth}{0}
% Redefines (sub)paragraphs to behave more like sections
\ifx\paragraph\undefined\else
\let\oldparagraph\paragraph
\renewcommand{\paragraph}[1]{\oldparagraph{#1}\mbox{}}
\fi
\ifx\subparagraph\undefined\else
\let\oldsubparagraph\subparagraph
\renewcommand{\subparagraph}[1]{\oldsubparagraph{#1}\mbox{}}
\fi

%%% Use protect on footnotes to avoid problems with footnotes in titles
\let\rmarkdownfootnote\footnote%
\def\footnote{\protect\rmarkdownfootnote}

%%% Change title format to be more compact
\usepackage{titling}

% Create subtitle command for use in maketitle
\providecommand{\subtitle}[1]{
  \posttitle{
    \begin{center}\large#1\end{center}
    }
}

\setlength{\droptitle}{-2em}

  \title{Lupus Symptoms on Twitter}
    \pretitle{\vspace{\droptitle}\centering\huge}
  \posttitle{\par}
    \author{Sabrina Kent}
    \preauthor{\centering\large\emph}
  \postauthor{\par}
    \date{}
    \predate{}\postdate{}
  

\begin{document}
\maketitle

\hypertarget{introduction}{%
\subsection{Introduction}\label{introduction}}

When studying immunology, I was intrigued by the way diseases like
rheumatoid arthritis and lupus work. Lupus, in particular, is
fascinating - not ony its mechanism of action, but all the factors
leading to difficulty in diagnosing patients correctly. A simple google
on lupus hits on the basics:

\begin{itemize}
\tightlist
\item
  It can take years for a patient to be diagnosed with lupus
\item
  Lupus can ``masquerade'' as other diseases and cause great frustration
  (and worse)
\item
  A huge part of the issue is the wide variety - and therefore the
  patient profile inconsistency - of symptoms
\end{itemize}

Digging deeper, I went to lupus.org and, from there, to their lupus
message board. Searching for posts about patients exhibiting unusual
symptoms led to something that floored me - according to many posts,
lupus patients struggle with getting their doctors to acknowledge that
their less typical symptoms are, indeed, related to lupus. However, many
of these same posts had an astounding number of responses from other
patients confirming that the ``atypical'' symptom being discussed was
actually experienced by many members of the online lupus community.

That seems like an important disconnection between what lupus patients
experience and what established medical knowledge says lupus does to a
patient.

I'd like very much to collect data on patient-reported symptoms
(specifically, symptoms that the patients feel are connected to their
lupus diagnosis) and connect that with medical papers on lupus. However,
data mining directly from a message board like the one I used to collect
anecdotal evidence of an interesting problem would not be ethical.

So I went to twitter with my favorite r package for this kind of work,
rtweet. (Kearney MW (2019). rtweet: Collecting Twitter Data. R package
version 0.6.9, \url{https://cran.r-project.org/package=rtweet}. )

My official starting question is: is Twitter a good source for data on
patient-reported lupus symptoms?

The more specific question, of course is about patient-reported lupus
symptoms that are not considered to be standard symptoms of lupus by the
medical community, but I'm going to start by casting a wide net and
doing an exploratory analysis.

\begin{Shaded}
\begin{Highlighting}[]
\KeywordTok{library}\NormalTok{(ggplot2)}
\KeywordTok{library}\NormalTok{(tidyr)}
\KeywordTok{library}\NormalTok{(rtweet)}
\KeywordTok{library}\NormalTok{(stringr)}
\KeywordTok{library}\NormalTok{(dplyr)}
\end{Highlighting}
\end{Shaded}

\begin{verbatim}
## 
## Attaching package: 'dplyr'
\end{verbatim}

\begin{verbatim}
## The following objects are masked from 'package:stats':
## 
##     filter, lag
\end{verbatim}

\begin{verbatim}
## The following objects are masked from 'package:base':
## 
##     intersect, setdiff, setequal, union
\end{verbatim}

\#\#Collect data from twitter

To download a data frame of tweets including ``lupus'' as a hashtag
requires a few minutes work with rtweet. Because of twitter's terms of
use, I won't be including the actual tweets I collected, but will be
able to share the resulting statistical information.

\begin{Shaded}
\begin{Highlighting}[]
\NormalTok{og_df <-}\StringTok{ }\KeywordTok{search_tweets}\NormalTok{(}
  \StringTok{"#lupus"}\NormalTok{, }\DataTypeTok{n =} \DecValTok{18000}\NormalTok{, }\DataTypeTok{include_rts =} \OtherTok{FALSE}
\NormalTok{)}
\end{Highlighting}
\end{Shaded}

\hypertarget{remove-unneeded-columns}{%
\subsection{Remove unneeded columns}\label{remove-unneeded-columns}}

Of the 90 variables downloaded automatically, only a few are potentially
useful. At this point I want to keep information on the actual tweet and
its hashtags, the account it came from, information on popularity - and
if there is an URL attached to the tweet.

\begin{Shaded}
\begin{Highlighting}[]
\NormalTok{lupus_df <-}\StringTok{ }\KeywordTok{select}\NormalTok{(og_df, screen_name, text, favorite_count, hashtags, urls_expanded_url)}
\end{Highlighting}
\end{Shaded}

\hypertarget{remove-tweets-from-organizations-linking-to-articles}{%
\subsection{Remove tweets from organizations linking to
articles}\label{remove-tweets-from-organizations-linking-to-articles}}

I kept the variable containing linked URLs, because many of the tweets I
first downloaded seemed to be generated by organizations and served as
advertisements back to articles on other websites about lupus. Those
tweets are not relevant here, and removing them will clean up my data
nicely.

I want to keep tweets that don't have links in them.

So I simply filtered my dataframe for only tweets for which the URL
variable is ``NA,'' easily increasing the proportion of tweets from
individuals rather than organizations and, particularly, removing tweets
that don't discuss lupus directly.

\begin{Shaded}
\begin{Highlighting}[]
\NormalTok{no_orgs <-}\StringTok{ }\KeywordTok{filter}\NormalTok{(lupus_df, }\KeywordTok{is.na}\NormalTok{(urls_expanded_url))}
\end{Highlighting}
\end{Shaded}

\hypertarget{create-new-columns}{%
\subsection{Create new columns}\label{create-new-columns}}

I'm going to start my exploration by testing for the proportion of my
tweets that contain interesting words:

\begin{itemize}
\item
  ``fibromyalgia'' is another difficult-to-diagnose disease that is
  often comorbid with lupus
\item
  ``flare'' is the term used by both patients and medical professionals
  for times when disease activity increases and more symptoms are seen
\item
  ``symptom'' - is the magic word!
\end{itemize}

\begin{Shaded}
\begin{Highlighting}[]
\NormalTok{no_orgs <-}\StringTok{ }\NormalTok{no_orgs }\OperatorTok
\StringTok{  }\KeywordTok{mutate}\NormalTok{(}\DataTypeTok{contains_flare =} \KeywordTok{grepl}\NormalTok{(}\StringTok{"flare"}\NormalTok{, text, }\DataTypeTok{fixed =} \OtherTok{TRUE}\NormalTok{))}

\NormalTok{no_orgs <-}\StringTok{ }\NormalTok{no_orgs }\OperatorTok
\StringTok{  }\KeywordTok{mutate}\NormalTok{(}\DataTypeTok{contains_fibro =} \KeywordTok{grepl}\NormalTok{(}\StringTok{"fibro"}\NormalTok{, text, }\DataTypeTok{fixed =} \OtherTok{TRUE}\NormalTok{))}

\NormalTok{no_orgs <-}\StringTok{ }\NormalTok{no_orgs }\OperatorTok
\StringTok{  }\KeywordTok{mutate}\NormalTok{(}\DataTypeTok{contains_symptom =} \KeywordTok{grepl}\NormalTok{(}\StringTok{"symptom"}\NormalTok{, text, }\DataTypeTok{fixed =} \OtherTok{TRUE}\NormalTok{))}
\end{Highlighting}
\end{Shaded}

\hypertarget{summarize-data}{%
\subsection{Summarize data}\label{summarize-data}}

What percent of tweets contain lupus + fibro, lupus + flare, lupus +
symptoms - fibro, lupus + flare + symptoms, lupus - flare + symptoms

\begin{Shaded}
\begin{Highlighting}[]
\NormalTok{num_flare =}\StringTok{ }\KeywordTok{sum}\NormalTok{(no_orgs}\OperatorTok{$}\NormalTok{contains_flare)}
\NormalTok{num_fibro =}\StringTok{ }\KeywordTok{sum}\NormalTok{(no_orgs}\OperatorTok{$}\NormalTok{contains_fibro)}
\NormalTok{num_symptom =}\StringTok{ }\KeywordTok{sum}\NormalTok{(no_orgs}\OperatorTok{$}\NormalTok{contains_symptom)}

\NormalTok{flare_and_symptom =}\StringTok{ }\KeywordTok{sum}\NormalTok{(no_orgs}\OperatorTok{$}\NormalTok{contains_symptom }\OperatorTok{&}\StringTok{ }\NormalTok{no_orgs}\OperatorTok{$}\NormalTok{contains_flare)}

\NormalTok{no_flare_yes_symptom =}\StringTok{ }\KeywordTok{sum}\NormalTok{(no_orgs}\OperatorTok{$}\NormalTok{contains_symptom }\OperatorTok{&}\StringTok{ }\OperatorTok{!}\NormalTok{no_orgs}\OperatorTok{$}\NormalTok{contains_flare)}

\NormalTok{num_flare}
\end{Highlighting}
\end{Shaded}

\begin{verbatim}
## [1] 17
\end{verbatim}

\begin{Shaded}
\begin{Highlighting}[]
\NormalTok{num_fibro}
\end{Highlighting}
\end{Shaded}

\begin{verbatim}
## [1] 24
\end{verbatim}

\begin{Shaded}
\begin{Highlighting}[]
\NormalTok{num_symptom}
\end{Highlighting}
\end{Shaded}

\begin{verbatim}
## [1] 14
\end{verbatim}

\begin{Shaded}
\begin{Highlighting}[]
\NormalTok{flare_and_symptom}
\end{Highlighting}
\end{Shaded}

\begin{verbatim}
## [1] 0
\end{verbatim}

\begin{Shaded}
\begin{Highlighting}[]
\NormalTok{no_flare_yes_symptom}
\end{Highlighting}
\end{Shaded}

\begin{verbatim}
## [1] 14
\end{verbatim}

So only 17 tweets in my data frame contain the word ``flare'' and only
19 contain ``symptom''? That's not very helpful - unless my dataframe is
somehow very tiny? I should have checked the dimensions immediately.
It's never too late!

\begin{Shaded}
\begin{Highlighting}[]
\KeywordTok{dim}\NormalTok{(og_df)}
\end{Highlighting}
\end{Shaded}

\begin{verbatim}
## [1] 1186   90
\end{verbatim}

\begin{Shaded}
\begin{Highlighting}[]
\KeywordTok{dim}\NormalTok{(no_orgs)}
\end{Highlighting}
\end{Shaded}

\begin{verbatim}
## [1] 522   8
\end{verbatim}

\begin{Shaded}
\begin{Highlighting}[]
\DecValTok{100} \OperatorTok{-}\StringTok{ }\KeywordTok{dim}\NormalTok{(no_orgs)}\OperatorTok{/}\KeywordTok{dim}\NormalTok{(og_df) }\OperatorTok{*}\StringTok{ }\DecValTok{100}
\end{Highlighting}
\end{Shaded}

\begin{verbatim}
## [1] 55.98651 91.11111
\end{verbatim}

So almost 60\% of my lupus-related tweets contained hyperlinks in them.

\textbf{Interesting Note:} Over half of the sampled lupus-related tweets
were from organizations, not individuals

That's not very encouraging as far as my original question, but it does
make sense. I collected a sampling of tweets based only on whether they
contained \#lupus. And twitter contains constant conversations of every
imaginable type. I should be expecting a small proportion of the
original dataset to contain tweets about symptoms.

Collecting a larger data set over time will increase the number of
tweets pertinent to my question, but I'm going to keep exploring to set
up a reusably analysis.

\hypertarget{examine-hashtags-in-tweet-collection}{%
\subsection{Examine hashtags in tweet
collection}\label{examine-hashtags-in-tweet-collection}}

My next step is to collect all the hashtags in the dataset and rank them
by frequency, creating a dataframe of tags.

\begin{Shaded}
\begin{Highlighting}[]
\CommentTok{#create a vector from every element in the hashtags column of no_orgs}
\NormalTok{all_tags <-}\StringTok{ }\KeywordTok{c}\NormalTok{(no_orgs}\OperatorTok{$}\NormalTok{hashtags, }\DataTypeTok{recursive =} \OtherTok{TRUE}\NormalTok{)}
\NormalTok{all_tags <-}\StringTok{ }\KeywordTok{tolower}\NormalTok{(all_tags)}

\CommentTok{#creates a vector of length all_tags with FALSE at }
\CommentTok{#indices of first occurrence of a tag, TRUE else}
\NormalTok{dups_index <-}\StringTok{ }\KeywordTok{duplicated}\NormalTok{(all_tags)}
\end{Highlighting}
\end{Shaded}

\begin{Shaded}
\begin{Highlighting}[]
\CommentTok{#create a dataframe holding each tag in column 1, number}
\CommentTok{#of times that tag occurs in twitter data in column 2}
\NormalTok{tag_freq_df <-}\StringTok{ }\KeywordTok{as.data.frame}\NormalTok{(}\KeywordTok{table}\NormalTok{(all_tags))}
\NormalTok{popular_tag_freq_df <-}\StringTok{ }\NormalTok{tag_freq_df[tag_freq_df}\OperatorTok{$}\NormalTok{Freq }\OperatorTok{>}\StringTok{ }\DecValTok{1}\NormalTok{, ]}

\CommentTok{#sort by tag frequency}
\NormalTok{popular_tag_freq_df <-}\StringTok{ }\NormalTok{popular_tag_freq_df }\OperatorTok
\StringTok{  }\KeywordTok{arrange}\NormalTok{(}\KeywordTok{desc}\NormalTok{(Freq)) }
\end{Highlighting}
\end{Shaded}

\begin{Shaded}
\begin{Highlighting}[]
\CommentTok{#the first row will be for the most frequent tag - lupus. Remove it.}
\NormalTok{popular_tag_freq_df <-}\StringTok{ }\NormalTok{popular_tag_freq_df[}\OperatorTok{-}\KeywordTok{c}\NormalTok{(}\DecValTok{1}\NormalTok{),]}
\KeywordTok{names}\NormalTok{(popular_tag_freq_df)}
\end{Highlighting}
\end{Shaded}

\begin{verbatim}
## [1] "all_tags" "Freq"
\end{verbatim}

\begin{Shaded}
\begin{Highlighting}[]
\KeywordTok{dim}\NormalTok{(popular_tag_freq_df)}
\end{Highlighting}
\end{Shaded}

\begin{verbatim}
## [1] 165   2
\end{verbatim}

I now have a data frame containing under 200 individual hashtags (after
removing \#lupus, of course), and the number of times each occurred.

\#\#Create a graph to show which hashtags are most popular

It's high time for some graphics. When creatting my data frame of
hashtags, I left out any tag that only occurred once. It will be very
interesting to see if most of the tags left occur many times, or if the
reverse is true - many tags occur relatively infrequently.

\begin{Shaded}
\begin{Highlighting}[]
\CommentTok{#visualize the frequencies of tags that occur more than once}
\KeywordTok{ggplot}\NormalTok{(popular_tag_freq_df, }\KeywordTok{aes}\NormalTok{(}\DataTypeTok{x =}\NormalTok{ all_tags, }\DataTypeTok{y =}\NormalTok{ Freq)) }\OperatorTok{+}
\StringTok{  }\KeywordTok{geom_boxplot}\NormalTok{()}
\end{Highlighting}
\end{Shaded}

\includegraphics{LupusSymptomsTwitter_files/figure-latex/unnamed-chunk-11-1.pdf}

That graph is much too dense to read - but is it useless? It does show
that:

\begin{itemize}
\tightlist
\item
  There are a LOT of tags
\item
  Many of these tags occur 10 times or less (in 10 or fewer tweets)
\item
  It would be helpful to know exactly how many tweets and how many tags
  we have now
\end{itemize}

\begin{Shaded}
\begin{Highlighting}[]
\CommentTok{#How many rows (each row representing a tweet) are in the processed source dataframe?}
\KeywordTok{dim}\NormalTok{(no_orgs)}
\end{Highlighting}
\end{Shaded}

\begin{verbatim}
## [1] 522   8
\end{verbatim}

\begin{Shaded}
\begin{Highlighting}[]
\CommentTok{#How many rows (each row representing a hashtag) are in our dataframe?}
\KeywordTok{dim}\NormalTok{(popular_tag_freq_df)}
\end{Highlighting}
\end{Shaded}

\begin{verbatim}
## [1] 165   2
\end{verbatim}

\begin{Shaded}
\begin{Highlighting}[]
\CommentTok{#If a tweet occurs 10 times (in 10 tweets), what percent of collected tweets is it appearing in?}
\DecValTok{100} \OperatorTok{*}\StringTok{ }\DecValTok{10}\OperatorTok{/}\KeywordTok{dim}\NormalTok{(popular_tag_freq_df)}
\end{Highlighting}
\end{Shaded}

\begin{verbatim}
## [1]   6.060606 500.000000
\end{verbatim}

Looking at that graph, we have a choice to make about our priorities -
do we want to see tweets that occur 10 times or more, or tweets that
occur 20 times or more (a stricter criterion would give us only tags
that are relatively very popular, but there are so few of those that we
might lose information)?

Adding horizontal lines at y = 10 and y = 20 to visualize this might
help.

\begin{Shaded}
\begin{Highlighting}[]
\KeywordTok{ggplot}\NormalTok{(popular_tag_freq_df, }\KeywordTok{aes}\NormalTok{(}\DataTypeTok{x =}\NormalTok{ all_tags, }\DataTypeTok{y =}\NormalTok{ Freq)) }\OperatorTok{+}
\StringTok{  }\KeywordTok{geom_boxplot}\NormalTok{() }\OperatorTok{+}\StringTok{ }\KeywordTok{geom_hline}\NormalTok{(}\DataTypeTok{yintercept =} \DecValTok{10}\NormalTok{) }\OperatorTok{+}\StringTok{ }\KeywordTok{geom_hline}\NormalTok{(}\DataTypeTok{yintercept =} \DecValTok{20}\NormalTok{)}
\end{Highlighting}
\end{Shaded}

\includegraphics{LupusSymptomsTwitter_files/figure-latex/unnamed-chunk-13-1.pdf}

Removing only tags that occur 20 or more times would show us the most
popular few, but there's a good chance those could be, by their
popularity, very generic:

\begin{Shaded}
\begin{Highlighting}[]
\CommentTok{#peeking at the top few tags}
\KeywordTok{head}\NormalTok{(popular_tag_freq_df)}
\end{Highlighting}
\end{Shaded}

\begin{verbatim}
##         all_tags Freq
## 2 chronicillness   53
## 3 lupusawareness   34
## 4        spoonie   31
## 5    chronicpain   28
## 6   fibromyalgia   28
## 7   lupuswarrior   20
\end{verbatim}

Yes. \#chronicillness, \#chronicpain, \#lupusawareness, and \#spoonie
are really only useful as other identifiers in the same way \#lupus is.
An argument could be made for discarding not only tweets with frequency
of \textless{}10, but also discarding those with frequency of
\textgreater{}20 for their over-generalizations.

However, \#fibromyalgia and \#antiphosphlipidsyndrome are autoimmune
diseases and may connect us to interesting tweets for questions of
dual-diagnosis or mistaken-diagnosis issues. Since this project is to
explore the nature of tweets containing \#lupus, it wouldn't be wise to
prune away possibly useful data.

Once again, the high variability in the specific subject/purpose of
tweets on lupus means that repeating this analysis with a much bigger
dataset would be very interesting.

Let's go back to the tag dataframe and remove any with a frequency lower
than 10.

\hypertarget{final-tag-selection}{%
\subsection{Final tag selection}\label{final-tag-selection}}

\begin{Shaded}
\begin{Highlighting}[]
\NormalTok{final_tag_freq_df <-}\StringTok{ }\NormalTok{popular_tag_freq_df[popular_tag_freq_df}\OperatorTok{$}\NormalTok{Freq }\OperatorTok{>=}\StringTok{ }\DecValTok{10}\NormalTok{, ]}
\KeywordTok{dim}\NormalTok{(final_tag_freq_df)}
\end{Highlighting}
\end{Shaded}

\begin{verbatim}
## [1] 22  2
\end{verbatim}

Our final data set contains only 30 tags. That's small enough to list
right here:

\begin{Shaded}
\begin{Highlighting}[]
\NormalTok{final_tag_freq_df}
\end{Highlighting}
\end{Shaded}

\begin{verbatim}
##                 all_tags Freq
## 2         chronicillness   53
## 3         lupusawareness   34
## 4                spoonie   31
## 5            chronicpain   28
## 6           fibromyalgia   28
## 7           lupuswarrior   20
## 8    rheumatoidarthritis   19
## 9                    sle   17
## 10     autoimmunedisease   16
## 11            autoimmune   13
## 12               anxiety   12
## 13 mentalhealthawareness   12
## 14               educate   11
## 15         endometriosis   11
## 16              migraine   11
## 17     multiplesclerosis   11
## 18                cancer   10
## 19                celiac   10
## 20               disease   10
## 21                   eds   10
## 22           lymedisease   10
## 23                  pots   10
\end{verbatim}

\hypertarget{conclusion}{%
\subsection{Conclusion}\label{conclusion}}

Examining these tags reveals some general patterns in the subject of the
connected tweets, but shows very few possible symptoms, much less
unusual symptoms. The final tag in this set, narcolepsy, is perhaps the
most interesting one. By going back to the tweet dataframe and selecting
only tweets using \#narcolepsy, I could read those tweets (there are
only ten, after all) and see if narcolepsy is being reported as a
lupus-related disorder.

\hypertarget{discussion}{%
\subsection{Discussion}\label{discussion}}

It's not surprising to discover that the wide variety in reasons people
tweet about lupus has made it difficult to find information on a very
specific question by scraping Twitter. However, beginning again with
rtweet's features for collecting tweets over time and building a much,
much larger dataset that way could, theoretically, help me. Current
computing resources don't allow this, but I do look forward to returning
to this data collection and analysis file.


\end{document}
